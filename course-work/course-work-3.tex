\section{Технологический раздел}
\label{sec:technology}

\subsection{Выбор технологий для реализации системы}

Были выбрали надежный и современный набор технологий для своего веб-сервиса «Бюро находок». Вот краткий обзор некоторых из наиболее важных технологий в вашем стеке и почему они полезны для проекта:

\begin{itemize}[wide=0pt]
	\item Next.js~--- это мощная платформа React, которая обеспечивает рендеринг на стороне сервера и генерацию статических сайтов, что может помочь улучшить производительность и SEO веб-приложения. Он также поддерживает маршруты API, что упрощает внутреннюю разработку в рамках одного проекта~\cite{bib:nextjs};
	
	\item React~--- популярная библиотека JavaScript для создания пользовательских интерфейсов, особенно одностраничных приложений, где требуется быстрый ответ на взаимодействие с пользователем. Это помогает в создании повторно используемых компонентов пользовательского интерфейса~\cite{bib:reactjs};
	
	\item tRPC позволяет создавать типобезопасные API без необходимости писать схемы или генерировать типы. Он использует TypeScript, чтобы обеспечить проверку типов входных и выходных данных API, что снижает вероятность ошибок во время выполнения и повышает производительность разработчиков~\cite{bib:trpc};
	
	\item Prisma~--- это ORM (реляционное сопоставление объектов), которое упрощает доступ к базе данных и обеспечивает безопасность типов. Адаптер Prisma для NextAuth.js упрощает создание надежных систем аутентификации, привязанных непосредственно к схеме вашей базы данных~\cite{bib:prisma};
	
	\item NextAuth.js~--- комплексное решение для аутентификации в приложениях Next.js. Он поддерживает различные стратегии входа в систему с минимальной настройкой, повышая безопасность приложения~\cite{bib:nextauth};
	
	\item React Query~--- это библиотека для эффективного извлечения, кэширования и обновления данных в приложениях React. Это помогает управлять состоянием сервера и синхронизировать пользовательский интерфейс с данными без использования бойлерплейт кода~\cite{bib:reactquery};
	
	\item Zustand~--- это минималистичное решение для управления состоянием, которое работает «из коробки» с React. Это просто, быстро и не требует шаблонного кода, как это делают другие библиотеки управления состоянием~\cite{bib:zustand};
	
	\item Tailwind CSS~--- это ориентированная на утилиты CSS-инфраструктура, наполненная классами, которые можно создавать для создания любого дизайна прямо в вашей разметке. Flowbite расширяет Tailwind дополнительными компонентами, ускоряя создание красивых и отзывчивых интерфейсов~\cite{bib:tailwind};
	
	\item AWS SDK позволяет легко использовать веб-сервисы Amazon, такие как S3 для хранения, SES для отправки электронной почты и т. д., прямо из приложения. Это имеет решающее значение для масштабирования и управления инфраструктурой~\cite{bib:awssdk};
	
	\item Headless UI~--- библиотека, которая предоставляет полностью неоформленные, полностью доступные компоненты пользовательского интерфейса, предназначенные для прекрасной интеграции с Tailwind CSS. Это помогает в создании пользовательских и доступных выпадающих списков, модальных окон и т.д~\cite{bib:headlessui}.
\end{itemize}

\subsection{Реализация модулей автоматизации процессов}

\subsubsection{Модуль регистрации и авторизации пользователей}

Модуль представляет реализацию OAuth с единовременной передачей данных пользователя.

OAuth (Open Authorization)~\cite{bib:oauth2}~--- это открытый стандарт авторизации, который позволяет пользователям предоставлять безопасный делегированный доступ к своим учетным записям на различных сервисах, не раскрывая свои пароли. Реализация клиента OAuth включает несколько ключевых этапов и может варьироваться в зависимости от используемой версии OAuth (например, OAuth 1.0 или OAuth 2.0). Страница логина представлена на~рис.~\ref{fig:login-page}.

\begin{figure}[htb]
	\centering
	\includegraphics[width=.95\textwidth]{images/login-page.png}
	\parskip=6pt
	\caption{Страница логина}
	\label{fig:login-page}
\end{figure}

Ниже описан процесс реализации клиента для OAuth 2.0.

\begin{enumerate}[wide=0pt]
	\item[1.] Перед тем как начать, вам нужно зарегистрировать ваше приложение на платформе, которую вы хотите использовать (например, Google, Facebook, GitHub). В процессе регистрации вам нужно будет указать:
	
	\begin{itemize}[wide=0pt]
		\item Название приложения;
		\item URL, на который будет перенаправлен пользователь после авторизации (callback URL);
	\end{itemize}

	После регистрации вы получите client ID и client secret, которые будут использоваться для аутентификации вашего приложения.
	
	\item[2.] Клиент должен перенаправить пользователя на URL авторизации, предоставленный сервисом, с необходимыми параметрами запроса:
	
	\begin{itemize}[wide=0pt]
		\item \textit{response\_type} (обычно \textit{code});
		\item \textit{client\_id};
		\item \textit{redirect\_uri};
		\item \textit{scope} (необязательно, зависит от того, к каким данным приложение стремится получить доступ);
		\item \textit{state} (рекомендуется для защиты от CSRF-атак);
	\end{itemize}
	
	Пользователь входит в систему (если еще не вошел) и подтверждает доступ к своим данным. После этого сервис перенаправляет пользователя обратно на \textit{redirect\_uri} с кодом авторизации в параметрах URL;
	
	\item[3.] После получения кода авторизации, приложение делает запрос на сервер сервиса для обмена кода на токен доступа. Этот запрос должен содержать:
	
	\begin{itemize}[wide=0pt]
		\item \textit{grant\_type} (обычно \textit{authorization\_code});
		\item \textit{code} (полученный код авторизации);
		\item \textit{redirect\_uri};
		\item \textit{client\_id};
		\item \textit{client\_secret};
		
	\end{itemize}
	
	Если запрос успешен, сервер ответит JSON-объектом, содержащим \textit{access\_token} (и возможно \textit{refresh\_token} и другие данные);
	
	\item[4.] Токен доступа используется для доступа к защищенным ресурсам пользователя. Он добавляется в заголовок HTTP-запроса как \textit{Authorization: Bearer <token>};
	
	\item[5.] Если получен \textit{refresh\_token}, его можно использовать для получения нового \textit{access\_token} после его истечения без необходимости повторной аутентификации пользователя.
	
\end{enumerate}


\subsubsection[Модуль\hspace*{10pt}бесконечных\hspace*{10pt}лент\hspace*{10pt}объявлений\hspace*{10pt}потерянных,\hspace*{10pt}найденных вещей]{Модуль бесконечных лент объявлений потерянных, найденных вещей}

Для создания бесконечной ленты объявлений в React можно воспользоваться следующим подходом, основанным на базовых принципах React и встроенных возможностях JavaScript:

\begin{itemize}[wide=0pt]
	\item создается состояние в компоненте, которое будет хранить массив объявлений и переменную для отслеживания, загружены ли все данные;
	
	\item реализуется функция, которая будет загружать порции данных (например, по 10 объявлений за раз). Эта функция должна обновлять ваше состояние, добавляя новые объявления к уже загруженным;
	
	\item добавляется обработчик события прокрутки к элементу, в котором отображаются объявления. Когда пользователь достигает конца списка, вызывайте функцию загрузки данных;
	
	\item необходимо убедиться, что новые данные загружаются только когда предыдущая загрузка завершена, чтобы избежать повторных запросов. Также нужно проверить, не достигнут ли конец списка данных, чтобы прекратить загрузку новых данных;
	
	\item используется метод map для преобразования массива объявлений в JSX-элементы, которые будут отображаться в пользовательском интерфейсе;
	
	\item используется техники оптимизации, такие как React.memo для компонентов объявлений, чтобы избежать ненужных ререндеров при добавлении новых объявлений;
	
	\item также нужно добавить обработку ошибок для ситуаций, когда загрузка данных может завершиться неудачей, например, при проблемах с сетью.
\end{itemize}

Реализация бесконечной ленты объявлений представлена на рис. \ref{fig:infinity-scroll-page}.

\begin{figure}[htb]
	\centering
	\includegraphics[width=.7\textwidth]{images/infinity-scroll-page.png}
	\parskip=6pt
	\caption{Страница бесконечной ленты объявлений}
	\label{fig:infinity-scroll-page}
\end{figure}

\subsubsection{Модуль добавления и поиска утерянных вещей}

Для добавления утерянных вещей реализована форма с вводом названия, описания, возможностью загружать картинки в объектное хранилище. Форма представлена на рисунке \ref{fig:lost-item-from-page}.

\begin{figure}[htb]
	\centering
	\includegraphics[width=.95\textwidth]{images/lost-item-from-page.png}
	\parskip=6pt
	\caption{Форма ввода утерянных вещей}
	\label{fig:lost-item-from-page}
\end{figure}

Полнотекстовый поиск по названиям и описаниям происходит по названиям и описанию объявлений. Реализация клиентской части поиска представлена на рисунке \ref{fig:search-page}.

\begin{figure}[htb]
	\centering
	\includegraphics[width=.95\textwidth]{images/search-page.png}
	\parskip=6pt
	\caption{Форма ввода утерянных вещей}
	\label{fig:search-page}
\end{figure}

\subsection*{Вывод по разделу}

Для обеспечения надежности приложения были внедрены механизмы логирования и мониторинга. Это позволяет оперативно выявлять и устранять возможные проблемы, а также анализировать поведение пользователей для дальнейшего улучшения приложения.

В итоге, все эти модули вместе обеспечивают безопасное и удобное функционирование приложения, соответствующее современным стандартам и требованиям.

Поиск утерянных вещей является актуальной проблемой, которая возникает при различных обстоятельствах. Эта проблема может возникнуть в результате потери ключей, документов, мобильных телефонов, кошельков или других ценных или важных вещей. В связи с этим существует необходимость разработки системы, которая поможет людям вернуть утерянные вещи.

Разработка системы для поиска утерянных вещей позволит создать удобный инструмент для поиска потерянных вещей, что приведет к уменьшению количества потерянных вещей и улучшению качества жизни людей. В ходе данной работы были проанализированы существующие системы и технологии, определены требования к разрабатываемой системе и ее функциональности. На основе этого анализа были разработаны и внедрены модули, обеспечивающие высокую безопасность, удобство использования и надежность системы.

Таким образом, итогом работы стало создание комплексного решения, которое решает проблему поиска утерянных вещей, улучшая повседневную жизнь пользователей.